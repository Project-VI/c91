\chapter{OpenStackのログに関する個人的な見解}
\setcounter{page}{1}

兎にも角にもログである。これは運用するにあたって一番(ではないかもしれんが)考慮する必要がある。これをまぁいいかって感じで運用をスタートさせてしまうと、あとで地獄を見るであろう(そのほうが後学のためでもあるという意見もある)。今回は、ボク個人が至極個人的な趣味趣向とミーハー精神に基づき構築し運用もしているログのシステムについて記す。たぶん、これからOpenStackを導入する人にとっては、参考になる部分は、はっきり言って無い。街談巷説、道聴塗説、話半分、そんな本章である。

\section{ログについて考えてみる}
前回出版した本 OpenStackの本 上級編ではこのように言及していた。

\begin{quotation}
	Q. 入れたはいいけど監視したいです
	A. 何を監視するか決めましょう
	まずは仮想マシンの監視ですが、これは普通のOS監視と同じ考え方で大丈夫です。ZabbixでもInfluxDBでも好きな物を使えますので、運用スタイルにあったものを選択しましょう。仮想環境ではマシンが頻繁に作成されては消えていく性質があるため、それに対応できるような監視ソリューションがいいと思います。

	コンピュートホストやネットワークノードなど、OpenStackで使われる物理サーバーの監視は普通のログやシステムメトリクスの監視に加えて、OpenStack専用の監視項目を追加する必要があるかもしれません。メモリは実使用メモリと仮想マシン割り当て量が大きく違ってきますので注意が必要です。また、ディスクもThinプロビジョニングですと、あるとき急にディスクがあふれる、なんていうこともあり得ます。libvirtのAPIなどを使用して(\verb|virsh|コマンドなど)仮想マシンホスト専用のメトリクスを取得することを考えるとよいと思います。
\end{quotation}

そう、何を監視するか、というのはとても重要である。死活監視は今回置いておくとして、監視対象のログは大きく2つに分けられる。1つ目は、イベントログ、2つ目は、パフォーマンスログである。イベントログというと、OpenStackの各コンポーネントから出力されたログである。主に、/var/log/XXX/配下にあるものを指す。パフォーマンスログとは、OpenStackの各コンポーネントを稼働させている各サーバ、及びKVMホストの性能データを指す。

これらを取得し、ストアし、分析し、障害対応のために役に立てるための定番の構成というものがある。

\begin{itemize}
	\item イベントログ (EFK, ELK stack, Splunk, etc)
	\item パフォーマンスログ (Collectd, Graphite, Grafana, etc)
\end{itemize}

\subsection{イベントログ}
まずは、イベントログに関して見ていこう。これについては、お金があるなら何も考えずにSplunkを導入しよう。あれは神ツールである。大規模にまで容易にスケールし、ログの構造を考えずにストアが可能で、他のツールと連携しアラートまで行えるのである。Splunkと同じことが出来る製品は他にない。Splunkを導入する予算がないのであれば、OpenStackの正確なイベントログ監視は諦めるべきである。ちなみに、経験上、Splunkそのものの運用には、ほぼ工数を割くことはない。Splunk導入は、まさにOpenStack監視における分水嶺的プロダクトといえる。

簡単な図にすると、こうなる。すべてのノードでSplunk Forwerderというデータコレクターを各OpenStackのホストにインストールする。そして、適切な設定をすれば、送りたいログを思うようにSplunkに転送してくれる。

\begin{lstlisting}
+-------------------------------------------+
|                                           |
|        OpenStack (Splunkfowerder)         |
|                                           |
+----+----------------+----------------+----+
     |                |                |
     |                |                |
     |                v                |
     |       +------------------+      |
     |       |                  |      |
     +------>|     Splunk       |<-----+
             |                  |
             +------------------+
\end{lstlisting}

なぜ、Splunk導入を強くすすめるかというと、それは、OpenStackのエラーログ(及びTRACEログ)の出力にある。OpenStackは、知っての通りPythonで記述されている。つまりどういうことかというと、OpenStackのコンポーネントが何かエラーを発生させてPythonの例外処理に入った際、ログはトレース(Pythonのコールスタック)付きのCRITICALなログメッセージの形式で出力される、ということである。

この例を見ていただければ、察しがつくと思うが、このようなログをそのままログ分析用のストレージに格納し分析まで可能にするプロダクトはSplunkだけである。

\begin{lstlisting}
  2013-02-25 21:05:51 17409 CRITICAL cinder [-] Bad or unexpected response from the storage volume backend API: volume group
   cinder-volumes doesn't exist
  2013-02-25 21:05:51 17409 TRACE cinder Traceback (most recent call last):
  2013-02-25 21:05:51 17409 TRACE cinder File "/usr/bin/cinder-volume", line 48, in <module>
  2013-02-25 21:05:51 17409 TRACE cinder service.wait()
  2013-02-25 21:05:51 17409 TRACE cinder File "/usr/lib/python2.7/dist-packages/cinder/service.py", line 422, in wait
  2013-02-25 21:05:51 17409 TRACE cinder _launcher.wait()
  2013-02-25 21:05:51 17409 TRACE cinder File "/usr/lib/python2.7/dist-packages/cinder/service.py", line 127, in wait
  2013-02-25 21:05:51 17409 TRACE cinder service.wait()
  2013-02-25 21:05:51 17409 TRACE cinder File "/usr/lib/python2.7/dist-packages/eventlet/greenthread.py", line 166, in wait
  2013-02-25 21:05:51 17409 TRACE cinder return self._exit_event.wait()
  2013-02-25 21:05:51 17409 TRACE cinder File "/usr/lib/python2.7/dist-packages/eventlet/event.py", line 116, in wait
  2013-02-25 21:05:51 17409 TRACE cinder return hubs.get_hub().switch()
  2013-02-25 21:05:51 17409 TRACE cinder File "/usr/lib/python2.7/dist-packages/eventlet/hubs/hub.py", line 177, in switch
  2013-02-25 21:05:51 17409 TRACE cinder return self.greenlet.switch()
  2013-02-25 21:05:51 17409 TRACE cinder File "/usr/lib/python2.7/dist-packages/eventlet/greenthread.py", line 192, in main
  2013-02-25 21:05:51 17409 TRACE cinder result = function(*args, **kwargs)
  2013-02-25 21:05:51 17409 TRACE cinder File "/usr/lib/python2.7/dist-packages/cinder/service.py", line 88, in run_server
  2013-02-25 21:05:51 17409 TRACE cinder server.start()
  2013-02-25 21:05:51 17409 TRACE cinder File "/usr/lib/python2.7/dist-packages/cinder/service.py", line 159, in start
  2013-02-25 21:05:51 17409 TRACE cinder self.manager.init_host()
  2013-02-25 21:05:51 17409 TRACE cinder File "/usr/lib/python2.7/dist-packages/cinder/volume/manager.py", line 95,
   in init_host
  2013-02-25 21:05:51 17409 TRACE cinder self.driver.check_for_setup_error()
  2013-02-25 21:05:51 17409 TRACE cinder File "/usr/lib/python2.7/dist-packages/cinder/volume/driver.py", line 116,
   in check_for_setup_error
  2013-02-25 21:05:51 17409 TRACE cinder raise exception.VolumeBackendAPIException(data=exception_message)
  2013-02-25 21:05:51 17409 TRACE cinder VolumeBackendAPIException: Bad or unexpected response from the storage volume
   backend API: volume group cinder-volumes doesn't exist
  2013-02-25 21:05:51 17409 TRACE cinder
\end{lstlisting}

Splunkを導入できない場合、この類のログを正確に分析することは諦めよう。考えるだけ時間の無駄である。せいぜい、FluentdのFilterを駆使しまくって頑張ってくれ。


\subsection{OpenStackホストのパフォーマンスログ}
OpenStackホストのパフォーマンスログの扱いは悩ましい。いわゆる、メトリクス監視と言うものなのだが、これは一度製品選定を行うとなかなか他のものに切り替えることができないからだ。大抵の企業は、GraphiteやCollectedやSensuを駆使して監視をしているであろう。が、OpenStackを運用するような企業は比較的インフラ規模が大きくなることを見越しているはずである。前述の監視ツール群でも事足りるであろうが、個人的には、より簡単にスケールアウトさせるためにKafkaをメッセージキューとして使うことを勧めたい。なぜなら、データを転送するクライアント(= KVMホストetc)が増加すると、確実にデータを格納するストレージとの間の通信が詰まるからだ。あと、最近流行ってるじゃん。

ということで、メトリクスに関するデータは、一旦Kafkaにためて、そこからFluentdでConsumerして、Time-Series DBに格納するというのがいいんじゃなかろうか。

データコレクターに何を使うかは考えものである。自分の中では以下の3つが考えられる。

\begin{itemize}
	\item Fluentd + dstat and df Plugin
	\item Telegraf
	\item Metricbeat
\end{itemize}

Fluentd + dstat and df Plugin はそれはそれできちんと動く。ただ、Fluentdはメトリクス監視を目的として作られているプロダクトではなく、公式でdstatやdfのPluginをサポートしているのではない。したがって、Fluentdの互換性に対してそれらのプラグインが追従するかどうか、という問題がある。現在のバージョンでは稼働するのだが、今後使い続けられるかどうかはわからない。そういった意味で、Fluentdによる監視はあまりおすすめしない。ちなみにこのPlugin問題はすべてのPluginに対して言うことが出来る。よって、Fluentdを導入する場合には、GitHub上のFluentプロジェクトで管理されている公式サポートがあるPluginのみを使うことを個人的にはおすすめする。

Telegrafは、InfluxDBに転送することを前提に設計されている。よって、通常メトリクスデータはInfluxDB Line Protocolの形式で転送される。以下のようなスタイルである。

\begin{lstlisting}
<measurement>[,<tag_key>=<tag_value>[,<tag_key>=<tag_value>]] <field_key>=<field_value>[,<field_key>=<field_value>] [<timestamp>]
\end{lstlisting}

これを、Kafkaに転送するのは良いが、Consumeが難しい。TelegrafでConsumeするのであれば、何も考えずにConsumeしてInfluxDBに投入するという流れにすることが出来るのだが、Telegraf自体は汎用性がなく、次章で少し説明するVMのデータのConsumeには向かない。よって、ConsumerはFluentd一択になるのだが、そうすると今度はInfluxDB Line Protocolのパースができなくなる。ただ、TelegrafからJsonでKafkaにデータを送ることが可能であるため、そこからかなり頑張ってFluentdでパースすれば、Telegrafを使うことも可能である。が、あまりおすすめできる方法ではない。

ということで、個人的なオススメはMetricbeatである。このプロダクトはもともとElasticsearchに転送することを想定して作られているためデータはすべてJson形式になっている。よって、Kafka ConsumerとしてFluentdを使用した場合でも、パースすることが容易となる。

\subsection{稼働しているVMのパフォーマンスログ}
OpenStackで稼働中のVMのメトリクススを取得する方法は、公式では無い。しかし、libvirtのAPIを使うことで簡単にVMのメトリクス監視が可能となる。その場合、データコレクターとしてはFluentdを使用すると良い。詳細については後述する。

\subsection{ログ転送の全体像}
大まかな全体像としてはこのような形になる。次の章からは1つずつ細かな説明をする。ここまで言及してなかったが、ログストレージとしては現時点でInfluxDB一択である。Elasticsearchは本来検索のためのものであり、時系列データを取り扱うのに最適とは言い難い。ただ問題は、InfluxDBが単一障害点になるということである(ただし、Enterprise版除く)。これを解決する手段としても、KafkaとConsumerで実現するダブルポスト方法がベターだと考えられる。

\begin{lstlisting}
+-------------------------------------------+
|                                           |
|      OpenStack (Metricbeat, Fluentd)      |
|                                           |
+----+----------------+----------------+----+
     |                |                |
     |                |                |
     |                v                |
     |       +------------------+      |
     |       |                  |      |
     +------>|      Kafka       |<-----+
             |                  |
             +--------+---------+
                      |
                      |
                      v
       +------------------------------+
       |                              |
       |   Kafka Consumer (Fluentd)   |
       |                              |
       +-+--------------------------+-+
         |                          |
         |                          |
         v                          v
+------------------+     +------------------+
|                  |     |                  |
|     InfluxDB     |     |     InfluxDB     |
|                  |     |                  |
+------------------+     +------------------+
\end{lstlisting}

\section{Fluentdについて}
前のセクションでSplunkを賞賛したが、Fluentdに関しては使い所を限定すれば、とても良いプロダクトである。Splunkは、データコレクターとしては汎用性は全く無く、Splunkにのみデータ転送が可能である。そういう意味で、Fluentdは、Splunk以外のログストレージ、その他の転送先(例えば、別のKafkaに転送するetc)にデータを転送する場合に、Pluginという形で対応が可能であるため、導入すればログシステム全体の拡張性が高くなる。ただし、前述のPluginの互換性問題には注意されたし。

前の章で、メトリクス監視にはMetricbeatがよいと言ったが、一応dstat + df Pluginを使った例を記す。まぁこれでも監視できるよって感じの例である。一旦Kafkaにデータを送って、それをまたFluentdでConsumeするという方法である。

\subsection{Kafka ProducerとしてのFluentdの設定}
dstatが入っていない場合は、事前にdstatをインストールしておく必要がある。またFluentdはv0.12以降をインストールする必要がある。インストールが必要なアプリとFluentdのPluginは以下の通りである。

\begin{itemize}
	\item Fluentd v0.12以降
	\item dstat
\end{itemize}

また、Fluentdに別途インストールが必要なPluginはこれである。夏辺りまでは、fluent-plugin-kafkaも別途インストールが必要だったが、最新版のtd-agentにはインストール時に同梱されているため別途入れる必要はない。がしかし、このKafka Pluginのバージョンは最新までアップデートしておくことをおすすめする。

\begin{itemize}
	\item fluent-plugin-record-reformer
	\item fluent-plugin-dstat
\end{itemize}

Fluentdのconfは以下のような設定にする。

\begin{lstlisting}
<source>
  @type dstat
  tag raw_dstat
  option -cmldrn
  delay 10
  tmp_file /tmp/dstat_all.csv
</source>

<source>
  @type exec
  tag raw_df
  command df -TP | sed 1d | sed 's/%//' | sed 's/\s\+/\t/g'
  run_interval 10s
  format tsv
  keys device,type,size,used,available,capacity,mounted_on
</source>

<filter raw_df>
  @type record_transformer
  enable_ruby true
  <record>
    hostname ${hostname}
  </record>
</filter>

<match raw_dstat raw_df>
  @type kafka_buffered
  brokers kafkabroker001:9092,kafkabroker002:9092
  default_topic metrics-topic
  flush_interval 60
  buffer_type file
  buffer_path /tmp/td-agent.*.buffer
  output_data_type json
  output_include_tag true
  output_include_time true
</match>
\end{lstlisting}

Brokerがkafkabroker001:9092,kafkabroker002:9092であり、Topic名がmetrics-topicという名前であるという仮定している。dfに関してはデバイスごとのキャパシティが取得できるのだが、Fluentdのtag名がdf.<デバイス名>になってしまい、かつ取得できるjsonにデバイス名が含まれていないため、\verb|reord_reformer|で書き換えている。

\subsection{Kafka ProducerとしてのFluentdの設定}
今回、FluentdをKafka Consuemerとしても使用する。ConsumerとしてKafkaからデータを取得してInfluxDBにダブルポストを行う。こっちにインストールが必要なFluentd Pluginはこちらである。

\begin{itemize}
	\item fluent-plugin-record-reformer
	\item fluent-plugin-influxdb
\end{itemize}

Fluentdのconfは以下のような設定にする。

\begin{lstlisting}
<source>
  @type kafka_group
  brokers kafkabroker001:9092,kafkabroker002:9092
  consumer_group consumer_group_001
  topics metrics-topic
  format json
</source>

<match metrics-topic>
  @type record_reformer
  tag ${record['tag']}
  enable_ruby true
  auto_typecast true
</match>

<match raw_dstat>
  @type copy
  <store>
    @type record_reformer
    tag cpu
    enable_ruby true
    auto_typecast true
    renew_record true
    <record>
      host   ${record['hostname']}
      usr    ${record['dstat']['total_cpu_usage']['usr'].to_f}
      sys    ${record['dstat']['total_cpu_usage']['sys'].to_f}
      idl    ${record['dstat']['total_cpu_usage']['idl'].to_f}
      wai    ${record['dstat']['total_cpu_usage']['wai'].to_f}
      hiq    ${record['dstat']['total_cpu_usage']['hiq'].to_f}
      siq    ${record['dstat']['total_cpu_usage']['siq'].to_f}
      time   ${record['time']}
    </record>
  </store>
  <store>
    @type record_reformer
    tag mem
    enable_ruby true
    auto_typecast true
    renew_record true
    <record>
      host   ${record['hostname']}
      used   ${record['dstat']['memory_usage']['used'].to_f}
      buff   ${record['dstat']['memory_usage']['buff'].to_f}
      cach   ${record['dstat']['memory_usage']['cach'].to_f}
      free   ${record['dstat']['memory_usage']['free'].to_f}
      time   ${record['time']}
    </record>
  </store>
  <store>
    @type record_reformer
    tag load
    enable_ruby true
    auto_typecast true
    renew_record true
    <record>
      host   ${record['hostname']}
      1m     ${record['dstat']['load_avg']['1m'].to_f}
      5m     ${record['dstat']['load_avg']['5m'].to_f}
      15m    ${record['dstat']['load_avg']['15m'].to_f}
      time   ${record['time']}
    </record>
  </store>
  <store>
    @type record_reformer
    tag disk
    enable_ruby true
    auto_typecast true
    renew_record true
    <record>
      host   ${record['hostname']}
      read   ${record['dstat']['dsk/total']['read'].to_f}
      writ   ${record['dstat']['dsk/total']['writ'].to_f}
      time   ${record['time']}
    </record>
  </store>
  <store>
    @type record_reformer
    tag diskio
    enable_ruby true
    auto_typecast true
    renew_record true
    <record>
      host   ${record['hostname']}
      read   ${record['dstat']['io/total']['read'].to_f}
      writ   ${record['dstat']['io/total']['writ'].to_f}
      time   ${record['time']}
    </record>
  </store>
  <store>
    @type record_reformer
    tag net
    enable_ruby true
    auto_typecast true
    renew_record true
    <record>
      host   ${record['hostname']}
      recv   ${record['dstat']['net/total']['recv'].to_f}
      send   ${record['dstat']['net/total']['send'].to_f}
      time   ${record['time']}
    </record>
  </store>
</match>

<match raw_df>
  @type record_reformer
  tag df
  enable_ruby true
  auto_typecast true
  renew_record true
  <record>
    host       ${record['hostname']}
    size       ${record['size'].to_f}
    used       ${record['used'].to_f}
    available  ${record['available'].to_f}
    capacity   ${record['capacity'].to_f}
    device     ${record['device']}
    type       ${record['type']}
    mounted_on ${record['mounted_on']}
    time       ${record['time']}
  </record>
</match>

<filter df cpu mem load disk diskio net>
  @type record_transformer
  renew_time_key time
</filter>

<filter df cpu mem load disk diskio net>
  @type record_transformer
  remove_keys time
</filter>

<match df cpu mem load disk diskio net>
  @type copy
  <store>
    @type influxdb
    host influxdb001
    port 8086
    dbname metrics_db
    user kafka
    password kafka
    use_ssl false
    verify_ssl false
    tag_keys ["host", "device"]
    time_precision s
    flush_interval 10s
  </store>
  <store>
    @type influxdb
    host influxdb002
    port 8086
    dbname metrics_db
    user kafka
    password kafka
    use_ssl false
    verify_ssl false
    tag_keys ["host", "device", "type", "mounted_on",]
    time_precision s
    flush_interval 10s
  </store>
</match>
\end{lstlisting}

InfluxDBのホスト名をinfluxdb001とinfluxdb002とし、DB名を両方とも\verb|metrics_db|と仮定している。また、InfluxDBのTagには、Strings型で取得しているhostとdeviceを設定している。


\section{InfluxDBについて}
\subsection{クラスタリング機能が無くなったOSS版InfluxDBで簡単な冗長性を持たせる}
OSS版InfluxDBについては元々クラスタリング機能が存在したが、v0.12以降はOSS版InfluxDBでのクラスタリングは廃止され、以後Influx Enterprise及び、Influx Cloudを通じての提供する方針になっている。

日本語の詳細記事はこちらを是非参照してみてほしい。

\begin{itemize}
	\item https://yakst.com/en/posts/3870
\end{itemize}

今回は、OSS版InfluxDBを使う前提なので、カジュアルにDBへのダブルポストを行い、かつ耐障害性もそこそこある構成にするためにKafkaを使っている。Kafkaを使うことで、メトリクスデータは、InfluxDBに投入される前の段階において、Retention期間中Kafkaクラスター内で冗長化された状態で保持される。また、Kafka Consumerとして使うFluentdもConsumer Groupを用いて複数台にスケールしている。そうすることで、監視対象のクライアントサイドからInfluxDBまでのどこで障害が起こっても、どこかの地点では確実にデータが保持され、SPOFになることはないはずである。


\subsection{InfluxDBを構築する}
2016年も終わりそろそろ2017年なるという時代なので、マニュアルで構築しようとかいう人はいないと信じてる。InfluxDBには、かなり出来の良いChefのレシピが転がっているので、それを使うのが妥当である。


\begin{itemize}
	\item https://github.com/bdangit/chef-influxdb
\end{itemize}

最近これを使ってv1.1.0までアップデートしたが、特に問題ない。最近のバージョンでも安定してデプロイが可能である。ちなみに、僕も少しだけこのレポジトリにコントリビュートしたというのは、ここだけの話である。

構築できたら、その後、最低限のDBの作成まではやっておく必要がある。Measurementsはデータ投入時に勝手に作成されるため何もしなくて問題ない。


\begin{lstlisting}
# Inlfuxdbのコンソールに入る
$$$ influx

# adminユーザを作成する
> CREATE USER admin WITH PASSWORD 'xxxxxxxx' WITH ALL PRIVILEGES

# adminユーザでログインし直すことが可能
$ influx -username admin -password xxxxxxxx

# metrics_dbというDBを作成(同時にRetention期間14日のポリシーshortを作成)
> CREATE DATABASE metrics_db WITH DURATION 14d REPLIDATION 1 SHARD DURATION 1d NAME short

# 必要ならnon-adminユーザも作っておく(以下は1例)
> CREATE USER kafkaconsumer WITH PASSWORD 'kafkaconsumer'
> CREATE USER grafana WITH PASSWORD 'grafana'
> CREATE USER openstackadmin WITH PASSWORD 'openstackadmin'
> GRANT ALL ON metrics_db TO kafkaconsumer
> GRANT READ ON metrics_db TO grafana
> GRANT ALL ON metrics_db TO openstackadmin
\end{lstlisting}

InfluxDBに関しては、これくらいである。特に運用も難しくなく、よしなに動いてくれる。負荷も低く、ディスク容量も食わない、唯一無二のTime-series DBである。

\section{OpenStack運用のためにチェックしたほうが良いイベントログについて}
イベントログについて言及すると、当たり前だがチェックしたほうがいいログはたくさんある。とりあえず、出力されてるログは全部集めよう。それは何かというと、/var/log/配下のすべてのコンポーネントに紐づくログのことである。

前述のとおり、イベントログはSplunkで分析するのがベターである。どういう検索分を使ってSplunk上でアラートやレポートやダッシュボードを作成すればいいかというところは各人によって異なってしまうので、ここで具体的な例を出すことはできない。作成したほうがいいアラートの一例をここに記す。

\begin{itemize}
	\item DHCPエラーチェック
	\item 全コンポーネントのログレベルERROR以上のログチェック
	\item APIのヘルスチェック
	\item 物理ディスクの障害検知
	\item 各プロジェクトごとのCinderボリューム使用率閾値チェック
	\item 各プロジェクトごとのNovaのvCPU使用率閾値チェック
	\item Glanceのスナップショット失敗ログチェック
	\item \verb|nf_conntrack_count|の値チェック
\end{itemize}

ダッシュボードの一例はこういったものがある。

\begin{itemize}
	\item 各ユーザごとの操作ログのモニタリング
	\item VM使用率のモニタリング
	\item CinderボリュームのQuota値モニタリング
	\item NovaのvCPU, MemoryなどのQuota値モニタリング
\end{itemize}

あくまで、一例である。ぶっちゃけると、こういったアラートやダッシュボードは、自分がOpenStackの運用を開始した時点では1つもなかった。ありとあらゆるトラブルに遭遇する中でちょっとづつ増えていったものである。

\section{KVMからVMのメトリクスを取得する方法}
nova-computeが稼働するKVMホストのサーバから、その上で動くインスタンスのメトリクスを取得する方法について記す。他の人がどうやってるのか、そもそもそんなことやっていないのか分からない。本サークスでは、各メトリクスの項目の情報をjsonで出力するPythonスクリプトをcronで実行してファイルにログを出力して、それをFluentdで読み込んで転送する、という流れで行っている。

\subsection{Pythonスクリプトの実装について}
メトリクス取得用のPythonスクリプトのリポジトリはこちらである。ちなみに、これを利用するには、Python 2.7とlibvirt Python bindingsが必要であるので、事前にインストールされたし。

\begin{description}
  \item[openstack-vm-stats-collector:] \url{https://github.com/Project-VI/openstack-vm-stats-collector}
\end{description}

ディレクトリ構造をツリーにするとこうなる。osvmstatというディレクトリ配下に各メトリクスごとのスクリプトを配置しており、かつ、その下のcommon配下に共通で使用する関数と例外処理を配置している。

\begin{lstlisting}
openstack-vm-stats-collector/
  |
  +-LICENSE
  |
  +-osvmstat/
  | |
  | +-block_info.py
  | |
  | +-block_stats.py
  | |
  | +-common/
  | | |
  | | +-__init__.py
  | | |
  | | +-exceptions.py
  | | |
  | | +-utils.py
  | |
  | +-interface_stats.py
  | |
  | +-memory_stats.py
  | |
  | +-vcpus.py
  |
  +-README.md

\end{lstlisting}

exceptions.pyは、特に珍しいことはしていないので解説しない。utils.pyから見ていこう。このライブラリでは、domain_xmlとnova_metadataという2つの関数を実装している。加えて、xml.etree.ElementTreeをimportしている。

domain_xmlは、xml.etree.ElementTreeを使いインスタンスIDの要素に紐づく情報を取得している。xml.etree.ElementTree自体は、任意のxml要素を渡すことでその要素の値を抽出することが出来るライブラリである。

\begin{lstlisting}
def domain_xml(domain):
    return xml.etree.ElementTree.fromstring(domain.XMLDesc())
\end{lstlisting}

\begin{lstlisting}
def nova_metadata(domain):
    metadata = xml.etree.ElementTree.fromstring(
        domain.metadata(libvirt.VIR_DOMAIN_METADATA_ELEMENT,
                        "http://openstack.org/xmlns/libvirt/nova/1.0"))
    return {
        "name": metadata.find("name").text,
        "project": {
            "uuid": metadata.find("owner/project").get("uuid"),
            "name": metadata.find("owner/project").text
            }}
\end{lstlisting}

\begin{lstlisting}
\end{lstlisting}

\begin{lstlisting}
\end{lstlisting}

\begin{lstlisting}
\end{lstlisting}

\begin{lstlisting}
\end{lstlisting}

\subsection{KVMホストで稼働するデータ転送用のFluentdについて}

\subsection{Kafka Consumerで稼働するConsume用のFluentdについて}



