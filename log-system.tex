\documentclass[9pt,b5paper,tombo,openany]{jsbook}

\usepackage[dvipdfmx]{graphicx}
\usepackage{listings}
\usepackage{inconsolata}
\usepackage{url}
\usepackage{titlesec}
\usepackage[dvipdfmx]{color}
\usepackage{setspace}
\usepackage{wrapfig}

\lstset{basicstyle={\small\ttfamily}}
\lstset{frame=single}
\lstset{breaklines=true}
\lstset{lineskip=-1.5pt}
\lstset{framesep=6pt}

\renewcommand{\kanjifamilydefault}{\gtdefault}
\renewcommand{\familydefault}{\sfdefault}
\renewcommand{\contentsname}{}
\renewcommand{\baselinestretch}{1.1}

\definecolor{gray75}{gray}{0.75}
\titleformat{\chapter}[hang]{\Huge\bfseries}{\thechapter\textcolor{gray75}{|}}{0pt}{\Huge\bfseries}

\begin{document}

\noindent
{\Huge OpenStackの本 番外編}

\vspace*{-1in}
\begin{minipage}{0.4\paperwidth}
	\tableofcontents
\end{minipage}

\vspace*{1in}
\begin{minipage}{0.4\paperwidth}
	\begin{spacing}{0.75}
	\end{spacing}
\end{minipage}

\thispagestyle{empty}

\chapter{OpenStackアップグレード超入門}

\setcounter{page}{1}

\marginpar{\includegraphics[width=30mm]{mini1.png}}本章は、 Qiita に投稿した OpenStack Advent Calendar 2015 の 12 月 3 日の記事である「 OpenStack をアップグレードしたら心臓止まりかけた話」という物語を改めて書き記した章である(ただし、改訂版というわけではない)。

\section{ログについて考えてみる}
\section{Fluentdについて}
\section{Kafkaについて}
\section{InfluxDBについて}
\section{OpenStack運用のためにチェックしたほうが良いログについて}
\section{KVMからVMのメトリクスを取得する方法}




\chapter{あとがき}

\setstretch{0.96}

\section*{こじろー}

\marginpar{\includegraphics[width=30mm]{mini3.png}}

\section*{まっきーさん}

\marginpar{\includegraphics[width=30mm]{mini3.png}}

\noindent
代わって檻には一匹の精悍な豹が入れられた。

\noindent
(西岡兄姉(2010)『カフカ・断食芸人』ヴィレッジブックス)

\vspace*{\stretch{1}}
\setstretch{1.0}
\begin{minipage}{0.5\paperwidth}
	著者:こじろー・まっきー

	挿絵:かとう

	発行:2016年12月31日

	印刷:POPLS (\url{http://www.inv.co.jp/~popls/})
\end{minipage}

\end{document}
